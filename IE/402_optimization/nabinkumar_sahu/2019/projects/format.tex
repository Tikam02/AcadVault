\documentclass[runningheads, a4paper, oribibl]{llncs}

\setcounter{tocdepth}{3}
\usepackage{graphicx}
\graphicspath{{../images/}}
\usepackage{epstopdf}
\usepackage{standalone}
\usepackage{xcolor}
\usepackage{tikz}
\usetikzlibrary{fit}
\usetikzlibrary{shapes,snakes,calc}

\usepackage{listings, color}
\definecolor{dkgreen}{rgb}{0,0.6,0}
\definecolor{gray}{rgb}{0.5,0.5,0.5}
\definecolor{mauve}{rgb}{0.58,0,0.82}
 \lstset{frame=tb,
language=Matlab,
aboveskip=3mm,
belowskip=3mm,
howstringspaces=false,
columns=flexible,
basicstyle={\small\ttfamily},
numbers=none,
numberstyle=\tiny\color{gray},
keywordstyle=\color{blue},
commentstyle=\color{dkgreen},
stringstyle=\color{black},
breaklines=true,
breakatwhitespace=false,
tabsize=2,
numbers=left,
numbersep=5pt,
title=\lstname
}
\usepackage[section]{placeins}
\usepackage{amsmath,amssymb, cancel}
\usepackage{url}
\urldef{\mailsa}\path|201601408@daiict.ac.in|
\urldef{\mailsb}\path|201601059@daiict.ac.in|
\newcommand{\keywords}[1]{\par\addvspace\baselineskip
\noindent\keywordname\enspace\ignorespaces#1}
\renewcommand\thesubsection{\thesection(\alph{subsection})}

\begin{document}

\mainmatter

\title{High Performance Computing Report}
\author{Nishi Doshi\\Roshani}
\institute{Dhirubhai Ambani Institute of Information and Communication Technology\\
\mailsa\\
\mailsb\\
}

\maketitle
\section{Implementation Details}
\subsection{Brief and clear description about the Serial implementation}
Hi
\subsection{Brief and clear description about the implementation of the approach (Parallelization Strategy, Mapping of computation to threads)}
Hello

\section{Complexity and Analysis Related}\subsection{Complexity of serial code}
Hi
\subsection{Complexity of parallel code (split as needed into work, step, etc.) }
Hi
\subsection{Cost of Parallel Algorithm}
Hi
\subsection{Theoretical Speedup (using asymptotic analysis, etc.)}
HI
\subsection{Estimated Serial Fraction }
HI
\subsection{Tight upper bound based on Amdahl's Law}
IH
\subsection{Number of memory accesses}
HI
\subsection{Number of computations}
HI


        \section{Curve Based Analysis}
\subsection{Time Curve related analysis (as no. of processor increases)}
HI
\subsection{Time Curve related analysis (as problem size increases, also for serial)}
HI
\subsection{Speedup Curve related analysis (as problem size and no. of processors increase)}
HI
\subsection{Efficiency Curve related analysis}
HI
\subsection{Karp-Flatt metric related analysis}
HI

\section{Further Detailed Analysis}
\subsection{Major serial and parallel overheads}
Hi
\subsection{Memory wall related analysis}
HI
\subsection{Cache coherence related analysis}
Hi
\subsection{False sharing related analysis}
HI
\subsection{Load balance analysis}
Hi
\subsection{Granularity related analysis}
HI
\subsection{Scalability related analysis}
HI

\section{Additional Approach Analysis}
\subsection{Analysis of any other concepts/factors you think were important in your problem-approach combination}
nk
\subsection{Further details (Code balance , machine balance analysis, how much of peak performance achieved in terms of \%)}
hi
\subsection{Advantages/Disadvantages of your approach}
hi
\subsection{Difficulties faced while implementing this approach}
hi
\subsection{Additional Comments}
hi

\section{Graphs}

\end{document}